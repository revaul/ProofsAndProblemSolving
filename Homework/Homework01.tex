% !TEX root = /Users/us2009801/Documents/GitHub/ProofsAndProblemSolving/main.tex
\chapter{Homework 1 - Saturday, November 3\ts{rd}, 2018}
\section{\S 1 Differential Equations}
\begin{prob}
Analyze the logical forms of the following statements. Use A to represent "Alice has a dog," B to represent "Bob has a dog," and C to represent "Carol has a cat" to write each as a symbolic statement.
\begin{enumerate}
\item Either Alice or Bob has a dog.\\
$A \vee B $
\item Neither Alice nor Bob has a dog, but Carol has a cat.\\
$ \neg (A \wedge B) \wedge C $
\item Either Alice has a dog and Carol has a cat, or Bob has a dog and Carol does not have a cat.\\
$ (A \wedge C) \vee (B \wedge \neg C) $
\end{enumerate}
This is similar to Example 1.1.2 and to Exercise 2 in Section 1.1 of your SNHU MAT299 textbook.
\end{prob}
\begin{prob}
If D stands for "Doug is tall" and E stands for "Edie is short," what English sentences are represented by the following expressions?
\begin{enumerate}
\item $(D \wedge E) \vee \neg D$\\
Either Doug is tall and Edie is short, or Doug is not tall
\item $(D \vee \neg E) \wedge \neg(D \wedge E)$\\
Either Doug is tall or Edie is not short and both Doug is not tall and Edie is not short
\item $\neg D \wedge ((E \wedge D) \vee \neg E )$\\
Doug is not tall and either Edie is short and Doug is tall, or Eddie is not short
\end{enumerate}
This is similar to Example 1.1.3 and to Exercise 6 in Section 1.1 of your SNHU MAT299 textbook.
\end{prob}
\begin{prob}
Make a truth table for the following formula.
$$ (G \vee \neg H) \wedge \neg(G \wedge L) $$
This is similar to Example 1.2.2 and to Exercise 2 in Section 1.2 of your SNHU MAT299 textbook.
\begin{center}
 \begin{tabular}{|c|c|c|c|c|c|c|c|} 
 \hline
 G & H & L & $\neg H$ & $G \vee \neg H$ & $G \wedge L$ & $\neg G \wedge L$ & $ (G \vee \neg H) \wedge \neg(G \wedge L) $ \\ 
 \hline
T & T & T & F & T & T & F & F\\
T & T & F & F & T & F & T & T\\
T & F & T & T & T & T & F & F\\
T & F & F & T & T & F & T & T\\
F & T & T & F & F & F & T & F\\
F & T & F & F & F & F & T & F\\
F & F & T & T & T & F & T & T\\
F & F & F & T & T & F & T & T\\
 \hline
\end{tabular}
\end{center}
\end{prob}
\begin{prob}
Use truth tables to determine which of the following formulas are equivalent to each other.
\begin{enumerate}
\item $ (J \wedge K) \vee ( \neg J \wedge \neg K) $ \\
\begin{center}
\begin{tabular}{|c|c|c|c|c|c|c|} 
\hline
J & K & $J \wedge K$ & $ \neg J $ & $ \neg K $ & $ \neg J \wedge \neg K $ & $ (J \wedge K) \vee ( \neg J \wedge \neg K) $ \\
\hline
T & T & T            & F          & F          & F                        & T                                             \\
T & F & F            & F          & T          & F                        & F                                             \\
F & T & F            & T          & F          & F                        & F                                             \\
F & F & F            & T          & T          & T                        & T                                            \\
\hline
\end{tabular}
\end{center}
\item $ J \vee K $ \\
\begin{center}
\begin{tabular}{|c|c|c|} 
\hline
J & K & $ J \vee K $ \\
\hline
T & T & T            \\
T & F & T            \\
F & T & T            \\
F & F & F           \\
\hline
\end{tabular}
\end{center}
\item $ J \wedge \neg K$\\
\begin{center}
\begin{tabular}{|c|c|c|c|} 
\hline
J & K & $ \neg K $ & $ J \wedge \neg K $ \\
\hline
T & T & F          & F                   \\
T & F & T          & T                   \\
F & T & F          & F                   \\
F & F & T          & F                   \\
\hline
\end{tabular}
\end{center}
\item $ \neg (\neg J \vee K) $\\
\begin{center}
\begin{tabular}{|c|c|c|c|c|} 
\hline
J & K & $ \neg J $ & $ \neg J \vee K $ & $ \neg ( \neg J \vee K ) $ \\
\hline
T & T & F          & T                 & F                          \\
T & F & F          & F                 & T                          \\
F & T & T          & T                 & F                          \\
F & F & T          & T                 & F                          \\
\hline
\end{tabular}
\end{center}
\item $ (J \wedge \neg K) \vee K $\\
\begin{center}
\begin{tabular}{|c|c|c|c|c|} 
\hline
J & K & $ \neg K $ & $ J \wedge \neg K $ & $ ( J \wedge \neg K ) \vee K $ \\
\hline
T & T & F          & F                   & T                              \\
T & F & T          & T                   & T                              \\
F & T & F          & F                   & T                              \\
F & F & T          & F                   & F                              \\
\hline
\end{tabular}
\end{center}
\end{enumerate}
This is similar to Example 1.2.4 and to Exercise 8 in Section 1.2 of your SNHU MAT299 textbook.
\end{prob}
\begin{prob}
Use truth tables to determine which of the following formulas are tautologies, which are contradictions, and which are neither.
\begin{enumerate}
\item $ (M \wedge \neg N) \vee ( \neg M \wedge N) $\\
\begin{center}
\begin{tabular}{|c|c|c|c|c|c|c|} 
\hline
M & N & $ \neg N $ & $ M \wedge \neg N $ & $ \neg M $ & $ \neg M \wedge N $ & $ (M \wedge \neg N) \vee ( \neg M \wedge N) $ \\
\hline
T & T & F          & T                   & F          & F                   & T                                             \\
T & F & T          & F                   & F          & F                   & F                                             \\
F & T & F          & F                   & T          & T                   & T                                             \\
F & F & T          & F                   & T          & F                   & F                                              \\
\hline
\end{tabular}
\end{center}
\item $ (M \wedge \neg N) \wedge ( \neg M \wedge N) $ \\
\begin{center}
\begin{tabular}{|c|c|c|c|c|c|c|} 
\hline
M & N & $ \neg N $ & $ M \wedge \neg N $ & $ \neg M $ & $ \neg M \wedge N $ & $ (M \wedge \neg N) \wedge ( \neg M \wedge N) $ \\
\hline
T & T & F          & T                   & F          & F                   & F                                               \\
T & F & T          & F                   & F          & F                   & F                                               \\
F & T & F          & F                   & T          & T                   & F                                               \\
F & F & T          & F                   & T          & F                   & F                                               \\
\hline
\end{tabular}
\end{center}
This is a Contradiction because the formula will always evaluate to false.
\item $ ( \neg M \wedge \neg N) \vee ( \neg M \vee N) \vee (M \wedge \neg N) $\\
\begin{center}
\begin{tabular}{|c|c|c|c|c|c|c|c|c|} 
\hline
M & N & $ \neg M $ & $ \neg N $ & $ \neg M \wedge \neg N $ & $ \neg M \vee N $ & $ M \wedge \neg N $ & \begin{tabular}{@{}c@{}}$ ( \neg M \wedge \neg N ) \vee $ \\ $ ( \neg M \vee N ) $ \end{tabular} & \begin{tabular}{@{}c@{}}$ ( \neg M \wedge \neg N) \vee $ \\ $  ( \neg M \vee N) \vee $ \\ $  (M \wedge \neg N) $ \end{tabular}  \\
\hline
T & T & F & F & F & T & F & T & T \\
T & F & F & T & F & F & T & F & T \\
F & T & T & F & F & T & F & T & T \\
F & F & T & T & T & T & F & T & T \\
\hline
\end{tabular}
\end{center}
\end{enumerate}
This is similar to Example 1.2.6 and to Exercise 9 in Section 1.2 of your SNHU MAT299 textbook.
\end{prob}
\begin{prob}
Use the laws stated in the text to find simpler formulas equivalent to these formulas. Explain the reasoning that you used to find your solution.
\begin{enumerate}
\item $ \neg ( \neg Q \vee ( \neg P \wedge Q) ) $
\item $ ((P \wedge Q) \wedge \neg R) \vee (P \wedge \neg (Q \vee R)) $
\end{enumerate}
This is similar to Examples 1.2.5, 1.2.6, and 1.2.7 and to Exercise 12 in Section 1.2 of your SNHU MAT299 textbook
\end{prob}
$\vee$ Or Vee\\
$\wedge$ and Wedge\\
$\neg \cup \vee \wedge$




\begin{center}
\begin{tabular}{|c|c|c|} 
\hline

\hline
 \\
\hline
\end{tabular}
\end{center}













































